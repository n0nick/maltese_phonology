\title{Stress Related Vowel Deletion in Maltese}
\author{
        \textsc{Sagie Maoz}\\
        Department of Linguistics\\
        Tel Aviv University\\
        % Tel Aviv, 69978, \underline{Israel}\\
		\\
        \normalsize
            \texttt{sagiemao@mail.tau.ac.il}
}
\date{
%	3\super{rd} Draft\\
	\today
}
\documentclass[12pt,draft]{article}

\usepackage[paper=a4paper,dvips,left=2cm,right=2cm,top=2.5cm]{geometry}

\usepackage{tabularx} % For tables width control
\usepackage{tipa} % IPA font type
% \usepackage{OTtablx} % OT Tablaux formats
% \usepackage{times} % Adobe Times as default font
\usepackage{pstricks}
\usepackage{colortab}
%\usepackage{pifont}

\usepackage{chicago} % Chicago Bibliography style

\usepackage{setspace}
\doublespacing

% count paragraphs
\setcounter{secnumdepth}{5}

\begin{document}
\maketitle

% And here we go.

\section{Introduction}

As a Semitic language, Maltese uses a very structured verb morphology to denote inflection on top of items from its stem lexicon.
This allows us to inspect the phenomenon of vowels being deleted from the stem in particular positions, and the interaction of such process with the Maltese stress system.
Namely, we would notice how adding different suffixes to a verb can change the production of the stem segments and the resulting stress of the final product word.

\subsection{Background}
Maltese is the national language of Malta. It is a Semitic language spoken by almost 400,000 people \cite{borg1997}.
Maltese is a descendant of the \emph{Siculo-Arabic} dialect developed in Sicily and Malta, later to be heavily influenced by Italian, Sicilian and English vocabularies.

\pagebreak

Below we list the Maltese phonetic inventory \cite{borg1997}:

\paragraph*{(1) Consonants}
\mbox{} % An empty paragraph

\newcolumntype{C}{>{\centering\arraybackslash}X} % C = centered X

\begin{table}[htdp]
%\caption{Consonants Inventory}
\begin{tabularx}{400pt}{|l||C C|C C|C C|C C|C C|C C|}
	\hline
	& & & & & \multicolumn{2}{c|}{(Post-)} & & & & & & \\
	&
	\multicolumn{2}{c|}{Labial} &
	\multicolumn{2}{c|}{Dental} &
	\multicolumn{2}{c|}{Alveolar} &
	\multicolumn{2}{c|}{Velar} &
	\multicolumn{2}{c|}{Palatal} &
	\multicolumn{2}{c|}{Laryngeal}\\\hline\hline
	
	Nasals &
	& m &
	& &
	& n &
	& &
	& &
	& \\\hline
	
	Stops &
	p & b &
	t & d &
	& &
	k & g &
	& &
	\textipa{P} & \\\hline
	
	Affricates &
	& &
	\textipa{\t{ts}} & \textipa{\t{dz}} &
	\textipa{\t{tS}} & \textipa{\t{dZ}} &
	& &
	& &
	& \\\hline
	
	Fricatives &
	f & v &
	s & z &
	\textipa{S} & &
	& &
	& &
	h & \\\hline
	
	Trills &
	& &
	& r &
	& &
	& &
	& &
	& \\\hline
	
	Approximants &
	& l &
	& &
	& &
	& w &
	& j &
	& \\\hline
\end{tabularx}
\end{table}

\paragraph*{(2) Vowels}

\subparagraph*{a. Monophtongs:}
\mbox{} % An empty paragraph

\begin{table}[htdp]
%\caption{Vowels Inventory}
\begin{tabularx}{200pt}{|l||C|C|C|C|C|C|}
	\hline
	&
	\multicolumn{2}{c|}{Front} &
	\multicolumn{2}{c|}{Central} &
	\multicolumn{2}{c|}{Back} \\\hline\hline
	
	High &
	\textipa{I} & \textipa{I:} \textipa{i:} &
	& &
	\textipa{U} & \textipa{U:} \\\hline
	
	Mid  &
	\textipa{E} & \textipa{E:} &
	& &
	\textipa{O} & \textipa{O:} \\\hline
	
	Low  &
	& &
	\textipa{5} & \textipa{5:} &
	& \\\hline
\end{tabularx}
\end{table}

\subparagraph*{b. Diphthongs:}
Seven diphthongs exist in Maltese:\\\
\textipa{/5U/},
\textipa{/5I/},
\textipa{/EU/},
\textipa{/EI/},
\textipa{/IU/},
\textipa{/OI/} and
\textipa{/OU/}.

\pagebreak

\subsubsection*{(3) Syllable structures\footnotemark[1]}

\begin{table}[htdp]
%\caption{Syllable structures}
\begin{tabularx}{\textwidth}{|l||l X|l X|l X|}
	\hline
	&
	\multicolumn{2}{c|}{Word Initial} &
	\multicolumn{2}{c|}{Word Medial} &
	\multicolumn{2}{c|}{Word final} \\\hline\hline
	
	V &
	\textipa{\underline{U}.nU:r} & 'honour' &
	\multicolumn{2}{c|}{---} &
	\multicolumn{2}{c|}{---} \\\hline
	
	CV &
	\textipa{\underline{kI}.tEp} & 'he wrote' &
	\textipa{m5h.\underline{mU}.\t{dZ}i:n} & 'dirty (pl.)' &
	\textipa{ip.\underline{kI}} & 'cry (Imp.)' \\\hline
	
	CCV &
	\textipa{\underline{dgE}.tsU} &	'to hoard (2pl.)' &
	\textipa{bI-\underline{PzI:}.PEs} & 'with pigs' &
	\textipa{dgE.\underline{tsU}} &	'to hoard (2pl.)' \\\hline
	
	CCCV &
	\textipa{\underline{ntr5}.m5}\footnotemark[2] & 'to assemble' &
	\textipa{dIs.tIn.\underline{tsjO}.nI} &	'distinc-tion' &
	\textipa{dgE.tsI.\underline{tsn5}} & 'to hoard (1pl.)' \\\hline
	
	VC &
	\textipa{\underline{ip}.kI} & 'cry (Imp.)' &
	\multicolumn{2}{c|}{---} &
	\multicolumn{2}{c|}{---} \\\hline
	
	CVC &
	\textipa{\underline{pEt}.nE} & 'comb' &
	\textipa{O.\underline{r5n}.\t{dZ}O} & 'orange' &
	\textipa{I:.\underline{bEs}} & 'hard' \\\hline
	
	CCVC &
	\textipa{\underline{tlIf}.t5} & 'I lost it (f.)' &
	\textipa{lIs.\underline{tr5m}.bE.ri:.ja} & 'the oddity' &
	\textipa{PO.rO.\underline{blOk}} & 'it (m.) has drawn nearer in time'\\\hline
	
	CCCVC &
	\textipa{\underline{sfrOn}.d5}\footnotemark[2] & 'to collapse' &
	\multicolumn{2}{c|}{---} &
	\multicolumn{2}{c|}{---} \\\hline
	
	VCC &
	\multicolumn{2}{c}{} &
	\textipa{\underline{E:nt}} &
	\multicolumn{3}{l|}{'I helped'} \\\hline
	
	CVCC &
	\multicolumn{2}{c|}{---} &
	\textipa{I.\underline{\t{tS}Ejn}.stOr} & 'the chainstore' &
	\textipa{we\t{dZ}.\underline{\t{dZ}5jt}} & 'I hurt (Inf-pl.)' \\\hline
	
	CCVCC &
	\multicolumn{2}{c}{} &
	\textipa{\underline{tl5pt}} &
	\multicolumn{3}{l|}{'I prayed'} \\\hline
	
	CCCVCC &
	\multicolumn{2}{c}{} &
	\textipa{\underline{str5ht}} &
	\multicolumn{3}{l|}{'I rested'} \\\hline
	
\end{tabularx}
\end{table}

\footnotetext[1]{Unless otherwise noted, examples taken from \citeN{borg1997}}
\footnotetext[2]{\citeN{mifsudloan1997}}

% Many more syllable structures are allowed; Borg-Azzopardi \cite{borg1997maltese} define the canonical syllable as \\(C)(C)(C)V(C)(C) for monosyllabic words and (C)(C)V(C) for multisyllabic words.

From the syllable inventory above, we can draw some generalisations on syllable structure:

\begin{itemize}

	\item Onset-less syllables are only allowed on word initial positions. -- $^*V]_{\sigma} $
	
	\item Only complex codas with 2 consonants are allowed word-finally. -- $^*CCC]_w$
	
%TODO	\item Word-final long vowels are prohibited. -- $^*V\textipa{:}]_w$
	
\end{itemize}

\clearpage

\subsubsection*{(4) Stress system}

Setting aside loan words, the original Maltese stress system is as follows \cite{wolf2012}:

\begin{description}
\item [(a)] Stress on the ultima, if it is superheavy (or the word is monosyllabic), else
\item [(b)] Stress on the penult, if it is heavy (or the word is bisyllabic), else
\item [(c)] Stress on the antepenult.
\end{description}

This is in fact the same stress system observed in Standard Arabic \cite{HALPERN09.16}.

\subsection{Theoretical Background}

\paragraph*{}
Let's consider the Maltese suffixes for the 2\super{nd} person singular subjects; the zero suffix \textsl{\textipa{+\O}} for 'he' and the suffix \textsl{\textipa{+Et}} for 'she' \cite{brame1974}. Using these, we can assume the UR for the following forms of the verbs \textsl{\textipa{h5t5f}} 'to grab' and \textsl{\textipa{bEz5P}} 'to spit':

\paragraph*{(5)} {\mbox{}}
\begin{table}[htdp]
\begin{tabularx}{400pt}{|l| X | X | X|}
	\hline
	&
	\multicolumn{1}{C|}{UR} &
	\multicolumn{1}{C|}{PR} &
	\multicolumn{1}{C|}{Gloss}\\\hline\hline
	
	(1a) &
	/\textipa{h5t5f+\O}/ &
	\textipa{"h5t5f} &
	'he grabbed' \\
	
	(1b) &
	/\textipa{h5t5f+Et}/ &
	\textipa{"h5tfEt} &
	'she grabbed' \\
	
%	(1c) &
%	/\textipa{h5t5f+t}/ &
%	\textipa{ht5ft} &
%	'I grabbed' \\
	\hline
	
	(2a) &
	/\textipa{bEz5P+\O}/ &
	\textipa{"bEz5P} &
	'he spit' \\
	
	(2b) &
	/\textipa{bEzaP+Et}/ &
	\textipa{"bEzPEt} &
	'she spit' \\
	
%	(2c) &
%	/\textipa{bEzaP+t}/ &
%	\textipa{bzaPt} &
%	'I spit' \\
	\hline
	
\end{tabularx}
\end{table}

It is immediately visible that the stem's original form is not preserved in forms (b) in the above examples. Specifically, the second vowel is deleted when a suffix is attached to the stem.

One would be tempted to suggest a straightforward explanation to the data, such as for example a vowel deletion rule on two-sided open syllables.
This kind of analysis, however, is bound to be challenged; first of all, we can easily find occurrences of open syllables in word medial positions, such as in \textsl{\textipa{bI-\underline{PzI:}.PEs}} 'with pigs', \textsl{\textipa{m5h.\underline{mU}.\t{dZ}i:n}} 'dirty (pl.)' and other examples from chart (3). It would seem the said rule would not account for such examples.

\pagebreak

An additional challenge with such analysis arises when reviewing the data for the \textsl{+t} suffix for the 1\super{st} person singular subject:

\paragraph*{(6)} {\mbox{}}
\begin{table}[htdp]
\begin{tabularx}{400pt}{|l| X | X | X|}
	\hline
	&
	\multicolumn{1}{C|}{UR} &
	\multicolumn{1}{C|}{PR} &
	\multicolumn{1}{C|}{Gloss}\\\hline\hline
	
	(1a) &
	/\textipa{h5t5f+\O}/ &
	\textipa{"h5t5f} &
	'he grabbed' \\
	
	(1b) &
	/\textipa{h5t5f+Et}/ &
	\textipa{"h5tfEt} &
	'she grabbed' \\
	
	(1c) &
	/\textipa{h5t5f+t}/ &
	\textipa{"ht5ft} &
	'I grabbed' \\
	
	\hline
	
	(2a) &
	/\textipa{bEz5P+\O}/ &
	\textipa{"bEz5P} &
	'he spit' \\
	
	(2b) &
	/\textipa{bEzaP+Et}/ &
	\textipa{"bEzPEt} &
	'she spit' \\
	
	(2c) &
	/\textipa{bEzaP+t}/ &
	\textipa{"bzaPt} &
	'I spit' \\
	\hline
	
\end{tabularx}
\end{table}

In the (c) forms above, we again notice a deletion of a stem vowel; but this time, it is the first stem vowel that is removed, to create a $CCVCC$ syllable.

This new data set forces us to rethink our analysis.
While it's possible to draw up a different rule to explain the deletion in (c), the two vowel deletion processes appear to have some common motivation, especially noticing that the deleted vowel is always un-stressed.

Thus, it would be ideal if we could formulate some unified system that explains these alternations, in relation to their motivation.
That is precisely the kind of issues that are better handled by Optimality Theory \cite{kager1999optimality} which makes use of constraints and their underlying order to explain phonological phenomenons and conspiracies \cite{kisseberth1970functional}.

\section {OT to the rescue}

\subsection {Unstressed vowel deletion}

\paragraph*{}
As mentioned, we speculate that the two vowel deletions we witnessed are related to the vowels originally being unstressed.
Therefore, let's formulate a constraint that would express this:

\begin{description}
	\item[(7) {\sc *V[-stress]}] - No unstressed vowels (syllables).
\end{description}

Looking at table (5), this constraint would indeed explain the deletions in examples (b), but alone it would fail to explain why the vowel wasn't deleted in examples (a), resulting for example in *[\textipa{\textsl{h5tf}}].
It's easy enough to think of a constraint that would avoid this form, though:

\begin{description}
	\item[(8) {\sc *ComplexCoda}] - Assign one violation-mark for every complex coda \cite{princesmolensky:1993cca}, i.e., a coda containing two or more consonants.
\end{description}

Let's add a rather obvious constraint that would account for the stress system always being present in PR.

\begin{description}
	\item[(9) {\sc Head(PrWd)}] - Each prosodic word has a unique head \cite[78]{mccarthy2002thematic} and therefore, it has stress.
\end{description}

Combining these three constraints we can now explain the \textsl{\textipa{+Et}} inflected form of the verb quite easily.

\paragraph*{(10)} {\mbox{}}
\begin{tabular}{|rrl||c|c|c|}\hline
\multicolumn{3}{|c||}{/\textipa{"h5t5f+Et}/} & {\sc Head(PrWd)} & {\sc *V[-stress]} & {\sc *CC} \\ \hline\hline
 a. &  & \textipa{"h5t5fEt} &  & **! & \\ \hline
 b. &  & \textipa{ht5fEt} & *! & * & \\ \hline
 c. & $\rightarrow$ & \textipa{"h5tfEt} &  & * & \\ \hline
 d. &  & \textipa{"h5t5ft} &  & * & *!\\ \hline
\end{tabular}
\\

We can identify in (10) that {\sc Head(PrWd)} dominates the two other constraints, because otherwise candidate (10b) would surely be selected.
However, there is no obvious order between {\sc *V[-stress]} and {\sc *CC}. Switching these two around would not affect the system's outcome. The ranking we have established, then, is described in (11):

\begin{description}
	\item[(11)] {{\sc Head(PrWd) >> *V[-stress], *CC}}
\end{description}

\subsection{No vowel deletion}

\paragraph*{}
Let's now analyse the \textsl{+\O} inflection in examples (a), where vowel deletion does not occur:
[\textsl{\textipa{h5t5f}}] does include an unstressed vowel, and so violates {\sc *V[-stress]}, and that would prove a domination of {\sc *CC} over {\sc *V[-stress]}.

\paragraph*{(12)} {\mbox{}}
\begin{tabular}{|rrl||c|c|c|c|}\hline
\multicolumn{3}{|c||}{/\textipa{h5t5f+\O}/} & {\sc Head(PrWd)} & {\sc *CC} & {\sc *V[-stress]} \\ \hline\hline
 a. & $\rightarrow$ & \textipa{"h5t5f} &   &  & *\\ \hline
 b. &  & \textipa{ht5f} & *! &  & \\ \hline
 c. &  & \textipa{"h5tf} &  & *! & \\ \hline
% d. &  & \textipa{"h5t5} &  & *! &  & \\ \hline
\end{tabular}
\\

Here in (12) we indeed witness that {\sc *CC} dominates {\sc *V[-stress]}. If we would have ordered these 2 constraints the other way around, then candidate (12a) would have been rejected in favour of candidate (12c), but that is not the case in PR.

So we have been able to confirm a tighter constraints ranking:

\begin{description}
	\item[(13)] {\sc Head(PrWd) >> *CC >> *V[-stress]}
\end{description}

\subsection{Stressed vowel deletion}

\paragraph*{}
The ranking we have established so far seems to be fairly straightforward, and explains the first two examples quite nicely.
Moving on to the third and final example, /\textsl{\textipa{h5t5f+t}}/, we are facing another constraint required for the system to be complete.

The \textsl{+t} suffix, when attached directly to the stem, necessarily creates a complex coda if the stem ends with a consonant.
In other words, {\sc*CC} is violated unless one of the stem segments is deleted.
Let's formulate a constraint to protect such stem segments from being deleted:

\begin{description}
 	\item[(14) {\sc MaxStem}] - Every segment of the input stem has a correspondent in the output stem \cite{chatzopoulos2008optimal}.
\end{description}

Generally, we might assume that {\sc MaxStem} has a high priority, to maintain the existence and semantics of stems in Maltese.

\pagebreak

We can now suggest a tableau explaining /\textsl{\textipa{h5t5f+t}}/:

\paragraph*{(15)} {}
\begin{tabular}{|rrl||c|c|c|c|}\hline
\multicolumn{3}{|c||}{/\textipa{h5t5f+t}/} & {\sc Head(PrWd)} & {\sc MaxStem} & {\sc *CC} & {\sc *V[-stress]} \\ \hline\hline
 a. & & \textipa{h5"t5ft} & & & * & *! \\ \hline
 b. & $\rightarrow$ & \textipa{"ht5ft} & & & * & \\ \hline
 c. & & \textipa{ht5ft} & *! & & * & \\ \hline
 d. & & \textipa{"h5tft} & & & **! & \\ \hline
 e. & & \textipa{h5"t5t} & & *! & & * \\ \hline
\end{tabular}
\\

Our constraints system is indeed able to reject candidates (15a), (15c), (15d) and (15e), selecting the remaining (15b) [\textipa{\textsl{"ht5ft}}]. This candidate, while violating {\sc *CC}, is still the optimal candidate in (15).

Notice that in order for tableau (15) to be valuable, we must conclude that the new constraint {\sc MaxStem} dominates both {\sc *CC} and {\sc *V[-stress]}. Otherwise, candidate (15e) would surely be selected.
That being said, we do not have any information about the ranking between {\sc Head(PrWd)} and {\sc MaxStem}, and they seem interchangeable.

\begin{description}
	\item[(16)] {\sc Head(PrWd), MaxStem >> *CC >> *V[-stress]}
\end{description}

\subsection{Possible vowel epenthesis}

\paragraph*{}
We have explained so far how Maltese chooses to preserve its stress system and syllable structures by deleting certain vowels when presented with morphological changes. Another obvious solution would have been the insertion of epenthetic vowels. To account for the lack of such insertions, we would have to consider the ranking of yet another constraint:

\begin{description}
	\item[(17) {\sc Dep-IO}] - Every segment of the output has a correspondent in the input. (Prohibits phonological epenthesis.) \cite{mccarthy1995faithfulness}
\end{description}

Re-analysing /\textsl{\textipa{h5t5f+t}}/ with the epenthesis option in mind, we must consider a new candidate (18f):

\paragraph*{(18)} {}
\begin{tabular}{|rrl||c|c|c|c|c|}\hline
\multicolumn{3}{|c||}{/\textipa{h5t5f+t}/} & {\sc Head(PrWd)} & {\sc MaxStem} & {\sc Dep-IO} & {\sc *CC} & {\sc *V[-stress]} \\ \hline\hline
% a. & & \textipa{h5"t5ft} & & & & * & *! \\ \hline
 b. & $\rightarrow$ & \textipa{"ht5ft} & & & & * & \\ \hline
% c. & & \textipa{ht5ft} & *! & & & * & \\ \hline
% d. & & \textipa{"h5tft} & & * & & **! & \\ \hline
% e. & & \textipa{h5"t5t} & & *! & & & * \\ \hline
 f. & & \textipa{h5"t5fEt} & & & *! & & * \\ \hline
\end{tabular}
\\

Without the new {\sc Dep-IO} constraint, candidate (18f) would only violate {\sc *V[-stress]} which we have already established is ranked lowest within the discussed constraints, and so it would be the optimal candidate.
This reflects the need to discuss {\sc Dep-IO} and its ranking. From tableau (18) we conclude that {\sc Dep-IO} necessarily dominates {\sc *CC} (otherwise, epenthesis would have been selected).
As in 2.3, we have no evidence for the ranking of {\sc Dep-IO} against {\sc Head(PrWd)} and {\sc MaxStem}, and from the analysed examples it seems the three are perfectly interchangeable.

Our constraints ranking is finalised, then, as in (19) below.

\begin{description}
	\item[(19)] {\sc Head(PrWd), MaxStem, Dep-IO >> *CC >> *V[-stress]}
\end{description}

\bibliographystyle{chicago}
\bibliography{maltese}

\end{document}


\paragraph*{(16)} {}
\begin{tabular}{|rrl||c|c|c|c|c|}\hline
\multicolumn{3}{|c||}{/\textipa{h5t5f+t}/} & {\sc Head(PrWd)} & {\sc MaxStem} & {\sc Dep-IO} & {\sc *CC} & {\sc *V[-stress]} \\ \hline\hline
 a. &  & \textipa{"h5tft} &  &  &  & **! & \\ \hline
 b. &  & \textipa{"h5t5fEt} &  &  & *! &  & *\\ \hline
 c. &  & \textipa{ht5fEt} & *! &  & * &  & **\\ \hline
 d. &  & \textipa{ht5ft} & *! &  &  & * & *\\ \hline
 e. &  & \textipa{h5t5t} &  & *! &  &  & \\ \hline
 f. & $\rightarrow$ & \textipa{h5t5ft} &  &  &  & * & *\\ \hline
\end{tabular}

Using the system defined so far, we would have assumed an output *[\textsl{\textipa{h5t5ft}}].
Yet the actual output turns out to be [\textsl{\textipa{ht5ft}}], which violates {\sc *CC} and is a result of deleting the first [\textsl{\textipa{5}}] vowel.
Taking into account rule (a) of the stress system in (4), we can conclude that the main stress in /\textsl{\textipa{h5t5f+t}}/ falls on the second syllable, and so {\sc *V[-stress]} is not violated in [\textsl{\textipa{ht5ft}}].
We are left with the question of ranking between {\sc *CC} and {\sc *V[-stress]}, and how it might affect /\textsl{\textipa{h5t5f+\O}}/ and /\textsl{\textipa{h5t5f+t}}/.

In addition, we notice that {\sc MaxStem} dominates both of these, because otherwise none of the constraints in (13) would have rejected candidate (13d).

% that we assume was originally the stressed vowel.
%\\

%In fact, that last assumption might be misleading; taking into account rule (a) of the stress system (See (4)), 

% ???? ????? ???
% DEP 
% *V[-stress] >> *CC