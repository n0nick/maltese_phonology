\documentclass[12pt,draft]{article}

\usepackage{tipa}
% \usepackage{OTtablx}

\title{Maltese}
\author{
        Sagie Maoz \\
                Department of Linguistics\\
        Tel Aviv University\\
        Tel Aviv, \underline{Israel}
}
\date{1st Draft\\\today}

\begin{document}
\maketitle

\section{Language}
Maltese is the national language of Malta. It is a Semitic language spoken by around 400,000 speakers\cite{Capdevila2004}.\\
Maltese is a descendant of the \emph{Siculo-Arabic} dialect developed in Sicily and Malta, later to be heavily influenced by Italian, Sicilian and English vocabularies.

\section{Data source} % sources*
\begin{itemize}
\item Azzopardi-Alexa. (1996). \emph{Maltese}. Routledge.
% \item Comrie, B. (2009). \emph{Introducing Maltese Linguistics: Selected Papers from the 1st International Conference on Maltese Linguistics, Bremen, 18-20 October, 2007}. John Benjamins Publishing.
\end{itemize}

\section{Phonetic inventory}

\subsection{Consonants}
\begin{tabular}{l||c c|c c|c c|c c|c c|c c|}
& & & & & \multicolumn{2}{c|}{Post-} & & & & & & \\
&
\multicolumn{2}{c|}{Labial} &
\multicolumn{2}{c|}{Dental} &
\multicolumn{2}{c|}{Alveolar} &
\multicolumn{2}{c|}{Velar} &
\multicolumn{2}{c|}{Pharyngeal} &
\multicolumn{2}{c|}{Glottal}\\\hline\hline
Nasals & & m  &  & & & n & & & & & & \\\hline
Stops & p & b & t & & & & k & & & & \textipa{P} & \\\hline
Affricates & & & \t{ts} & \t{dz} & \t{t\textipa{S}} & \t{d\textipa{Z}} & & & & & &\\\hline
Fricatives & f & v & s & z & \textipa{S} & & & & \textcrh & & & \\\hline
Trills & & & & r & & & & & & & & \\\hline
Approximants & & l & & & & & & & & & & \\\hline
\end{tabular}

\subsection{Vowels}
\begin{tabular}{l||c|c|c|c|c|c|}
&
\multicolumn{2}{c|}{Front} &
\multicolumn{2}{c|}{Central} &
\multicolumn{2}{c|}{Back} \\\hline\hline
High & \textipa{I} & \textipa{I:} \textipa{i:} & & & \textipa{U} & \textipa{U:} \\\hline
Mid  & \textipa{E} & \textipa{E:} & & & \textipa{O} & \textipa{O:} \\\hline
Low  & \textipa{6} & \textipa{6:} & & & & \\\hline
\end{tabular}

\subsection{Diphthongs}
Seven diphthongs exist in Maltese:
/\textipa{6I}/,
/\textipa{EI}/,
/\textipa{6U}/,
/\textipa{EU}/,
/\textipa{IU}/,
/\textipa{OI}/ and
/\textipa{OU}/.

\section{Syllable inventory}
TODO

\section{Generalisations}
TODO

\bibliographystyle{abbrv}
\bibliography{maltese}

\end{document}

TODO:
1. Text about diphthongs
2. a "high front lax long vowel""?
3. Syllable inventory
4. Generalisations
5. Make sure vowels are correct
6. Hide bibliography?

Abstract code:

\begin{abstract}
This is the paper's abstract \ldots
\end{abstract}
