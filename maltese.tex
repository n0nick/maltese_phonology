\title{Maltese}
\author{
        \textsc{Sagie Maoz}\\
        Department of Linguistics\\
        Tel Aviv University\\
        % Tel Aviv, 69978, \underline{Israel}\\
		\\
        \normalsize
            \texttt{sagiemao@mail.tau.ac.il}
}
\date{
	2st Draft\\
	\today
}
\documentclass[11pt,draft]{article}

\usepackage[paper=a4paper,dvips,left=2cm,right=2cm,top=2.5cm]{geometry}

\usepackage{tabularx} % For tables width control
\usepackage{tipa} % IPA font type
% \usepackage{OTtablx} % OT Tablaux formats
% \usepackage{times} % Adobe Times as default font

\begin{document}
\maketitle

% And here we go.

\section{Language}
Maltese is the national language of Malta. It is a Semitic language spoken by almost 400,000 people\cite{Azzopardi-Alexa1996}.\\
Maltese is a descendant of the \emph{Siculo-Arabic} dialect developed in Sicily and Malta, later to be heavily influenced by Italian, Sicilian and English vocabularies.

\section{Phonetic inventory}

\subsection{Consonants}

\newcolumntype{C}{>{\centering\arraybackslash}X} % C = centered X

\begin{table}[htdp]
%\caption{Consonants Inventory}
\begin{tabularx}{400pt}{|l||C C|C C|C C|C C|C C|C C|}
	\hline
	& & & & & \multicolumn{2}{c|}{Post-} & & & & & & \\
	&
	\multicolumn{2}{c|}{Labial} &
	\multicolumn{2}{c|}{Dental} &
	\multicolumn{2}{c|}{Alveolar} &
	\multicolumn{2}{c|}{Velar} &
	\multicolumn{2}{c|}{Palatal} &
	\multicolumn{2}{c|}{Laryngeal}\\\hline\hline
	
	Nasals &
	& m &
	& &
	& n &
	& &
	& &
	& \\\hline
	
	Stops &
	p & b &
	t & d &
	& &
	k & g &
	& &
	\textipa{P} & \\\hline
	
	Affricates &
	& &
	\textipa{\t{ts}} & \textipa{\t{dz}} &
	\textipa{\t{tS}} & \textipa{\t{dZ}} &
	& &
	& &
	& \\\hline
	
	Fricatives &
	f & v &
	s & z &
	\textipa{S} & &
	& &
	& &
	h & \\\hline
	
	Trills &
	& &
	& r &
	& &
	& &
	& &
	& \\\hline
	
	Approximants &
	& l &
	& &
	& &
	& &
	& j &
	& \\\hline
\end{tabularx}
\end{table}

Additionally, the voiced labial-velar approximant \textipa{/w/}.

\pagebreak

\subsection{Vowels}

\subsubsection{Monophtongs}

\begin{table}[htdp]
%\caption{Vowels Inventory}
\begin{tabularx}{200pt}{|l||C|C|C|C|C|C|}
	\hline
	&
	\multicolumn{2}{c|}{Front} &
	\multicolumn{2}{c|}{Central} &
	\multicolumn{2}{c|}{Back} \\\hline\hline
	
	High &
	\textipa{I} & \textipa{I:} \textipa{i:} &
	& &
	\textipa{U} & \textipa{U:} \\\hline
	
	Mid  &
	\textipa{E} & \textipa{E:} &
	& &
	\textipa{O} & \textipa{O:} \\\hline
	
	Low  &
	& &
	\textipa{5} & \textipa{5:} &
	& \\\hline
\end{tabularx}
\end{table}

I will normally use \textipa{/a/, /e/, /i/, /o/, /u/} to denote \textipa{/5/, /E/, /I/, /O/, /U/}, respectively.

\subsubsection{Diphthongs}
Seven diphthongs exist in Maltese:
\textipa{/5U/},
\textipa{/5I/},
\textipa{/EU/},
\textipa{/EI/},
\textipa{/IU/},
\textipa{/OI/} and
\textipa{/OU/}.

\section{Syllable inventory}

\begin{table}[htdp]
%\caption{Syllable Inventory}
\begin{tabularx}{\textwidth}{|l||l X|l X|l X|}
	\hline
	&
	\multicolumn{2}{c|}{Word Initial} &
	\multicolumn{2}{c|}{Word Medial} &
	\multicolumn{2}{c|}{Word final} \\\hline\hline
	
	V &
	\textipa{\underline{U}.nU:r} & 'honour' &
	\multicolumn{2}{c|}{---} &
	\multicolumn{2}{c|}{---} \\\hline
	
	CV &
	\textipa{\underline{kI}.tEp} & 'he wrote' &
	\textipa{m5h.\underline{mU}.\t{dZ}i:n} & 'dirty (pl.)' &
	\textipa{ip.\underline{kI}} & 'cry (Imp.)' \\\hline
	
	VC &
	\textipa{\underline{ip}.kI} & 'cry (Imp.)' &
	\multicolumn{2}{c|}{---} &
	\multicolumn{2}{c|}{---} \\\hline
	
	CVC &
	\textipa{\underline{pEt}.nE} & 'comb' &
	\textipa{O.\underline{r5n}.\t{dZ}O} & 'orange' &
	\textipa{I:.\underline{bEs}} & 'hard' \\\hline
	
	CV\textipa{:} &
	\textipa{\underline{d5:}.r5} & 'her house' &
	\textipa{bIP.\underline{zI:}.PEs} & 'with pigs' &
	\textipa{5\t{tS}.\t{tS}Et.\underline{t5:}} & 'he accepted it (f.)' \\\hline
	
	CCVC &
	\textipa{\underline{tlIf}.t5} & 'I lost it (f.)' &
	\textipa{5\t{tS}.\underline{\t{tS}Et}.t5} & 'he accepted' &
	\textipa{PO.rO.\underline{blOk}} & 'it (m.) has drawn nearer in time'\\\hline
	
\end{tabularx}
\end{table}

Many more syllable structures are allowed; Borg-Azzopardi \cite{borg1997maltese} define the canonical syllable as \\(C)(C)(C)V(C)(C) for monosyllabic words and (C)(C)V(C) for multisyllabic words.

\subsection{Generalizations}

\begin{itemize}

	\item Vowel-only $V$ syllables are only allowed on word initial positions. -- $^*\sigma[V]_\sigma $
	
	\item Only permitted word-final clusters are $CC$. -- $^*CCC]_w$
	
%TODO	\item Word-final long vowels are prohibited. -- $^*V\textipa{:}]_w$
	
\end{itemize}

\bibliographystyle{acm}
\bibliography{maltese}

\section*{References}

Borg, A., and M. Azzopardi-Alexander. 1997. \emph{Maltese}. New York: Routledge.

\end{document}

TODO:
1. More syllable structures

CODES:
Bibliography:
\bibliographystyle{abbrv}
\bibliography{maltese}

Abstract:
\begin{abstract}
This is the paper's abstract \ldots
\end{abstract}
