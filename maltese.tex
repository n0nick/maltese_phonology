\title{Maltese}
\author{
        \textsc{Sagie Maoz}\\
        Department of Linguistics\\
        Tel Aviv University\\
        % Tel Aviv, 69978, \underline{Israel}\\
		\\
        \normalsize
            \texttt{sagiemao@mail.tau.ac.il}
}
\date{
	2st Draft\\
	\today
}
\documentclass[11pt,draft]{article}

\usepackage[paper=a4paper,dvips,left=2cm,right=2cm,top=2.5cm]{geometry}

\usepackage{tabularx} % For tables width control
\usepackage{tipa} % IPA font type
% \usepackage{OTtablx} % OT Tablaux formats
% \usepackage{times} % Adobe Times as default font

% count paragraphs
\setcounter{secnumdepth}{5}

\begin{document}
\maketitle

% And here we go.

\section{Introduction}

\subsection{@@@}
@@@

\subsection{Language}
Maltese is the national language of Malta. It is a Semitic language spoken by almost 400,000 people\cite{borg1997maltese}.\\
Maltese is a descendant of the \emph{Siculo-Arabic} dialect developed in Sicily and Malta, later to be heavily influenced by Italian, Sicilian and English vocabularies.

\subsubsection{Phonetic inventory}

\paragraph{Consonants}
:

\newcolumntype{C}{>{\centering\arraybackslash}X} % C = centered X

\begin{table}[htdp]
%\caption{Consonants Inventory}
\begin{tabularx}{400pt}{|l||C C|C C|C C|C C|C C|C C|}
	\hline
	& & & & & \multicolumn{2}{c|}{(Post-)} & & & & & & \\
	&
	\multicolumn{2}{c|}{Labial} &
	\multicolumn{2}{c|}{Dental} &
	\multicolumn{2}{c|}{Alveolar} &
	\multicolumn{2}{c|}{Velar} &
	\multicolumn{2}{c|}{Palatal} &
	\multicolumn{2}{c|}{Laryngeal}\\\hline\hline
	
	Nasals &
	& m &
	& &
	& n &
	& &
	& &
	& \\\hline
	
	Stops &
	p & b &
	t & d &
	& &
	k & g &
	& &
	\textipa{P} & \\\hline
	
	Affricates &
	& &
	\textipa{\t{ts}} & \textipa{\t{dz}} &
	\textipa{\t{tS}} & \textipa{\t{dZ}} &
	& &
	& &
	& \\\hline
	
	Fricatives &
	f & v &
	s & z &
	\textipa{S} & &
	& &
	& &
	h & \\\hline
	
	Trills &
	& &
	& r &
	& &
	& &
	& &
	& \\\hline
	
	Approximants &
	& l &
	& &
	& &
	& &
	& j &
	& \\\hline
\end{tabularx}
\end{table}

Additionally, the voiced labial-velar approximant \textipa{/w/}.

\pagebreak

\paragraph{Vowels}

\subparagraph{Monophtongs}

:
\begin{table}[htdp]
%\caption{Vowels Inventory}
\begin{tabularx}{200pt}{|l||C|C|C|C|C|C|}
	\hline
	&
	\multicolumn{2}{c|}{Front} &
	\multicolumn{2}{c|}{Central} &
	\multicolumn{2}{c|}{Back} \\\hline\hline
	
	High &
	\textipa{I} & \textipa{I:} \textipa{i:} &
	& &
	\textipa{U} & \textipa{U:} \\\hline
	
	Mid  &
	\textipa{E} & \textipa{E:} &
	& &
	\textipa{O} & \textipa{O:} \\\hline
	
	Low  &
	& &
	\textipa{5} & \textipa{5:} &
	& \\\hline
\end{tabularx}
\end{table}

\subparagraph{Diphthongs:}
Seven diphthongs exist in Maltese:\\\
\textipa{/5U/},
\textipa{/5I/},
\textipa{/EU/},
\textipa{/EI/},
\textipa{/IU/},
\textipa{/OI/} and
\textipa{/OU/}.

\subsubsection{Syllable inventory}

\begin{table}[htdp]
%\caption{Syllable Inventory}
\begin{tabularx}{\textwidth}{|l||l X|l X|l X|}
	\hline
	&
	\multicolumn{2}{c|}{Word Initial} &
	\multicolumn{2}{c|}{Word Medial} &
	\multicolumn{2}{c|}{Word final} \\\hline\hline
	
	V &
	\textipa{\underline{U}.nU:r} & 'honour' &
	\multicolumn{2}{c|}{---} &
	\multicolumn{2}{c|}{---} \\\hline
	
	CV &
	\textipa{\underline{kI}.tEp} & 'he wrote' &
	\textipa{m5h.\underline{mU}.\t{dZ}i:n} & 'dirty (pl.)' &
	\textipa{ip.\underline{kI}} & 'cry (Imp.)' \\\hline
	
	CCV &
	\textipa{\underline{dgE}.tsU} &	'to hoard (2pl.)' &
	\textipa{bI-\underline{PzI:}.PEs} & 'with pigs' &
	\textipa{dgE.\underline{tsU}} &	'to hoard (2pl.)' \\\hline
	
	CCCV &
	\textipa{\underline{ptr5:}.vU} & 'with a beam' &
	\textipa{dIs.tIn.\underline{tsjO:}.nI} &	'distinction' &
	\textipa{dgE.tsI.\underline{tsn5}} & 'to hoard (1pl.)' \\\hline
	
	VC &
	\textipa{\underline{ip}.kI} & 'cry (Imp.)' &
	\multicolumn{2}{c|}{---} &
	\multicolumn{2}{c|}{---} \\\hline
	
	CVC &
	\textipa{\underline{pEt}.nE} & 'comb' &
	\textipa{O.\underline{r5n}.\t{dZ}O} & 'orange' &
	\textipa{I:.\underline{bEs}} & 'hard' \\\hline
	
	CCVC &
	\textipa{\underline{tlIf}.t5} & 'I lost it (f.)' &
	\textipa{lIs.\underline{tr5m}.bE.ri:.ja} & 'the oddity' &
	\textipa{PO.rO.\underline{blOk}} & 'it (m.) has drawn nearer in time'\\\hline
	
	CCCVC &
	\textipa{\underline{sfrOn}.d5} & 'to collapse' &
	\multicolumn{2}{c|}{---} &
	\multicolumn{2}{c|}{---} \\\hline
	
	VCC &
	\multicolumn{2}{c}{} &
	\textipa{\underline{E:nt}} &
	\multicolumn{3}{l|}{'I helped'} \\\hline
	
	CVCC &
	\multicolumn{2}{c|}{---} &
	\textipa{I.\underline{\t{tS}Ejn}.stOr} & 'the chainstore' &
	\textipa{we\t{dZ}.\underline{\t{dZ}5jt}} & 'I hurt (Inf-pl.)' \\\hline
	
	CCVCC &
	\multicolumn{2}{c}{} &
	\textipa{\underline{tl5pt}} &
	\multicolumn{3}{l|}{'I prayed'} \\\hline
	
	CCCVCC &
	\multicolumn{2}{c}{} &
	\textipa{\underline{str5ht}} &
	\multicolumn{3}{l|}{'I rested'} \\\hline
	
\end{tabularx}
\end{table}

% Many more syllable structures are allowed; Borg-Azzopardi \cite{borg1997maltese} define the canonical syllable as \\(C)(C)(C)V(C)(C) for monosyllabic words and (C)(C)V(C) for multisyllabic words.

\subsubsection{Generalizations}

\begin{itemize}

	\item Onset-less syllables are only allowed on word initial positions. -- $^*V]_{\sigma}\sigma $
	
	\item Only permitted word-final clusters are $CC$. -- $^*CCC]_w$
	
%TODO	\item Word-final long vowels are prohibited. -- $^*V\textipa{:}]_w$
	
\end{itemize}

\section*{References}

Borg, A., and M. Azzopardi-Alexander. 1997. \emph{Maltese}. New York: Routledge.

\end{document}

TODO:
1. More syllable structures

CODES:
Bibliography:
\bibliographystyle{acm} % acmtrans??
\bibliography{maltese}

Abstract:
\begin{abstract}
This is the paper's abstract \ldots
\end{abstract}
